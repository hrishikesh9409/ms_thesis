Disaster management pertains to coping with disaster, relief and evacuation in the event of a catastrophe. It is of paramount importance as it pertains to the protection of lives and property during the time of calamity. To mitigate the damage of such incidents, simulations are performed in order to formulate an efficient and optimized exit strategy for the people who are struck by such unfortunate incidents and to evacuate them to a safe zone as quickly as possible. The simulation of an evacuation of the participating agents broadly falls under three categories - Macro Agent Simulation, Micro Agent Simulation and/or a combination of Macro-Micro Agent Simulation (grouping according to age, gender, social ties such as family, friendships etc.). Although extensive work has been carried out to simulate disaster scenarios comprising of a massive number of agents with an aggregate set of characteristics (Macro Agent Simulation), very little work has been done so far to simulate realistic social behaviour of agents during a disaster scenario, especially pertaining to micro agent simulations. Realistic human and social behaviour characteristics are possible only through micro agent-based simulations as complex psychological and sociological paradigms can be mapped and hence dynamic real-time strategy decisions can be better understood, especially during a crisis. Through this thesis, I aim to present the simulation and comparison of various micro and macro agent scenarios.

In order to run the aforementioned simulations, an open source microscopic pedestrian simulator - PedSim is used. This tool not only simulates the various complex scenarios. it also provides visual feedback in real time. PedSim is a crowd simulation library capable of analysis of real time pedestrian flow rate. This agent-based model (ABM) tool is a class of computational models for simulating the actions and interactions of autonomous agents (either individual or collective entities such as organizations or groups) with a view to assessing their effects on the system as a whole. It combines elements of game theory, complex systems, emergency, computational sociology, multi-agent systems, and evolutionary programming. Although there are many proposed agent modelling simulators that are available, many if not most are with commercial license and do not support real time data flow. This tool is also customized and further extended in order to take an optimized routing algorithm for agents. The PedSim simulator is modelled for both indoor and outdoor areas (parking lots, forests etc.). The simulation tool is able to take sensory based data and apply them to the modelling agents/nodes to simulate real/design time analysis. The main advantage of this library is the architecture that enables visibility of users live using tcp/stream-based output through batch processing. The implementation is pure in C++ with minimal external dependencies like Qt Framework. The output of PedSim can be translated using a graphics engine to provide visually appealing realistic render of a walking person. The code is modular, scalable and open source available under GPL license. 

The main goal of this thesis is to exhibit an objective performance analysis and a comparison between an existing macro agent simulation algorithm against an optimized algorithm coupled with realistic constraints, testing them against a massively populated macro and micro agent model for real-time dynamic evacuation.