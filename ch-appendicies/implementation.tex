\chapter{Scenario Definition\label{ch:Scenario Definition}}

This section is used to provide a brief technical description of the implementation details that are involved whilst defining a scenario for the simulation of pedastrian agents within the PedSim environment. The library modules are written entirely in C++ whilst the topology description is described within a .xml file. The simulation tool parses the xml file and graphically projects the topology onto the simulation environment for visual reference.
The agent properties are also defined in the xml file such as their base speed, spawn location, total number of agents etc.

The section below provides an example of writing a topology description in the .xml file.

\section{Defining Topology}

The code snippet presented below depicts an example of defining the outline of the building, which was the topology used to run the simulations in. The dimensions of the building are strictly modeled according to the figure \ref{Coppito0 layout}. The dimensions have been scaled up to 5 times in order to suit the PedSim environment. A typical obstacle/wall description is defined as follows:
\\

\begin{lstlisting}[language=xml]
  <!-- Outer Walls -->
  
  <!-- TOP SIDE -->
  <obstacle x1="-120" y1="-90" x2="-7.5" y2="-90" />
  <obstacle x1="-7.5" y1="-90" x2="-7.5" y2="-60" />
  <obstacle x1="-7.5" y1="-60" x2="7.5" y2="-60" />
  <obstacle x1="7.5" y1="-60" x2="9.5" y2="-60" />
  <obstacle x1="17.5" y1="-60" x2="30" y2="-60" />
  <obstacle x1="30" y1="-60" x2="30" y2="-90" />
  <obstacle x1="30" y1="-90" x2="120" y2="-90" />

  <!-- RIGHT SIDE -->
  <obstacle x1="120" y1="-90" x2="120" y2="-15" />
  <obstacle x1="120" y1="-15" x2="105.5" y2="-15" />
  <obstacle x1="97.5" y1="-15" x2="97.5" y2="75" />

  <!-- BOTTOM SIDE -->
  <obstacle x1="97.5" y1="75" x2="7.5" y2="75" />
  <obstacle x1="7.5" y1="75" x2="4.5" y2="75" />
  <obstacle x1="-4.5" y1="75" x2="-7.5" y2="75" />
  <obstacle x1="-7.5" y1="75" x2="-97.5" y2="75" />
  <obstacle x1="-97.5" y1="75" x2="-97.5" y2="90" />
  <obstacle x1="-97.5" y1="90" x2="-135" y2="90" />


  <!-- LEFT SIDE -->
  <obstacle x1="-135" y1="90" x2="-135" y2="-30" />
  <obstacle x1="-135" y1="-30" x2="-112.5" y2="-30" />
  <obstacle x1="-120" y1="-38" x2="-120" y2="-90" />

\end{lstlisting}

From the above code snippet, the x and y pair points are given for drawing straight lines.
Partial and full agent pathing can be provided for the agents in order for them to start moving towards exit points during an emergency scenario.

\begin{lstlisting}[language=xml]

  <!-- waypoints are defined before agents  -->
  
  <waypoint id="w1" x="-75.0" y="-10" r="2" />
  <waypoint id="w2" x="-61" y="-3" r="2" />
  <waypoint id="w3" x="-50" y="-3" r="2" />
  <waypoint id="w4" x="-40" y="-3" r="2" />
  <waypoint id="w5" x="-30" y="-3" r="2" />
  <waypoint id="w6" x="-7.5" y="-3" r="2" />
  <waypoint id="w7" x="0" y="0" r="2" />
  <waypoint id="w8" x="0" y="30" r="2" />
  <waypoint id="w9" x="0" y="40" r="2" />
  <waypoint id="w10" x="0" y="50" r="2" />
  <waypoint id="w11" x="0" y="60" r="2" />
  <waypoint id="w12" x="0" y="70" r="2" />
  <waypoint id="w13" x="0" y="90" r="2" />

  \end{lstlisting}

The waypoint definition has coordianates x, y and radius r. The .xml file was written by referring the official documentation as provided in \cite{ref26}. Sample scenarios and tutorials regarding the same are also provided in the same documentation.

The final step in completing the scenario description is to include the agent properties. Given below is a sample agent definition in xml:

\begin{lstlisting}[language=xml]
  <agent x="30" y="-30" n="10" dx="50" dy="50">
  
  \end{lstlisting}

Where,
\begin{itemize}
\item $n$ is the number of agents.
\item $x$ and $y$ are the spawn location of the agents within the topography of the environment.
  \item dx and dy are used to determine the spread of the agents spawned. Agents are distributed evenly according to $x - dx$ and $x + dx$ and correspondingly along $y - dy$ and $y + dy$.
  \end{itemize}

Varying $n$ allows for defining the number of agents in a given group. This is particularly useful for the group simulation scenarios. In order to perform micro agent simulations, the $n$ is typically valued as 1. To summarize the agent definition, waypoints for the agents are defined in order to provide a more definitive pathing for the agents based on the spawn locations.

For a micro agent simulation, the agent definition will be as follows:
\\

\begin{lstlisting}[language=xml]
  
  <agent x="30" y="-30" n="1" dx="50" dy="60">
  <addwaypoint id="wu" />
  <addwaypoint id="wd" />
  <addwaypoint id="wz" />
  </agent> 
  <agent x="30" y="-30" n="1" dx="25" dy="30">
  <addwaypoint id="wu" />
  <addwaypoint id="wd" />
  <addwaypoint id="wz" />
  </agent> 
  <agent x="30" y="-30" n="1" dx="94" dy="100">
  <addwaypoint id="wu" />
  <addwaypoint id="wd" />
  <addwaypoint id="wz" />
  </agent> 
  <agent x="30" y="-30" n="1" dx="107" dy="112">
  <addwaypoint id="wu" />
  <addwaypoint id="wd" />
  <addwaypoint id="wz" />
  </agent>
    
\end{lstlisting}

Once the scenario to be modeled is described in the .xml file as given in the above code snippets, the file is then parsed by the PedSim library modules. Once the data is parsed from the .xml file, the agents are spawned according to the provided definitions and then PedSim proceeds to simulate the agents along the general waypoints as configured from the previous file.

The PedSim library modules are modified and extended in order to simulate the agents according to the algorithm present in \cite{ref5}. The technical details and description are provided in brief in the next section.
 
