\section{Pre-Modern Disaster Preparedness, Management and Relief Techniques}
\label{sec:pastwork:Pre-Modern Disaster Preparedness, Management and Relief Techniques}

Although disasters have existed throughout the history of the universe for all living organisms, human beings are perhaps the only one to be affected on a massive scale as a collective group as we are vulnerable and weak compared to other animals that live and co-exist with us. Hence human beings have taken many steps over the years fortifying and defending ourselves as best as we can through the various buildings and escape routes we design. The early inhabitants of mankind were not idle and did not become easy victims. There is historical evidence that the early man took various measures in order to cope with, reduce and mitigate risks. The mere fact that they chose to inhabit caves are a testament to this theory \cite{ref6}.

The ancient man also used other natural techniques for disaster preparedness. One of the prominent ones include the observation of animals since they are very perceptive to natural disasters, animal behavior can and has been used for the better part of the millenia to predict the onset of a natural disaster such as floods, earthquakes etc. One of the earliest works to be done in this field involves in the observation of the behavior of fish, specifically the catfish, as they were found to exhibit a definitive behavior in advance of the occurence of an earthquake \cite{ref7}. Other experiments and observations pertaining to such natural responses include the obeservation of ground electric field effects on behavior of Albino rats, Mongolian gerbils (sand rats), hair-footed Djungarian hamsters, guinea pigs, and red sparrows \cite{ref8}. A summary of such animal behavior and work is presented by Neeti Bhargava et. al. \cite{ref9}.

To combat various natural disasters that affect us, mankind has tried for over a millenia to improve and adapt buildings, and entire cities to cope during the time of a catatrosphe. There are two broad categories to the mitigation of these disasters that have been long employed \cite{ref10}:

\begin{itemize}
  \item structural based mitigation
  \item non-structural based mitigation
\end{itemize} 

Since my work pertains to indoor simulation and evacuation management, the discussion for non-structural mitigation forms here is considered unnecessary and beyond the scope of this thesis. Structural mitigation involves the presence of a building or a man-made mechanical or technological adaptations performed to reduce hazards and mitigate risks. The following sub-section expands more on the more modern techniques that have been developed over the course of the modern era and their implications. Although better and state of the art systems have been developed over the years, these systems have complications of their own. Failure of the perfect working of an evacuation system leads to the possible death of all the agents who are in need of evacuation. 
