\section{Simulation Tools and Constraints}
\label{sec:pastwork:Simulation Tool and Constraints}

In order to accomodate for accurate results and model precise human behavior, agent based modeling and simulations have become increasingly popular as computational expenses became cheaper and ever more accessible. In addition to developing state of the art routing algorithms and other obstacle avoidance mechanisms, it was also absolutely important to understand human behavior and the complex decisions that agents make during a catastrophic event. To understand the implications of human behavior and complex real-time decision making and other realistic constraints, this thesis presents various scenarios that are depicted in a small to a large scale microscopic agent based fashion in hopes to draw out some interesting conclusions based on the obtained results. One such example of the kind of work that we are hoping to perform is the impairment of certain agents based on the circumstances that are met with in real time during the disaster event. The work presented by Selain Kasereka et al. [20] offers a unique insight into modeling agents for a fire disaster and simulating the case for some interesting results. It should be mentioned that the case presented in the current thesis however is modeled for a generic scenario. In the work presented, the smoke and fire can affect the agent, thus reducing their potency and their ability to escape from danger. This damage to potency can be essentially translated to a reduced movement speed and a severe impairment to their cognitive ability to perform real time decisions. 

Another really interesting thing to point out is about psychology and human behavior during ground zero scenarios. Human panic and confusion is generally a chaotic element that must certainly be infused into such simulations as they provide further insight into human behavior and particularly decisions that lead to increase or the decrease in congestion. Based on certain decisions and paths, it may cause discomfort or physical pain to some and many agents within the promixity of such an incident. Ashutosh Trivedi et al. [21] provides a detailed analysis and where different strategies are evaluated and the corresponding evacuation time and physical discomfort caused to the agents are observed. He justifies the effect of social forces based on cohesion which, in my thesis, is also considered whilst perfoming simulation for microscopic agents. Our work revolves very closely the work done by A. Trivedi et al. as it also deals with social forces, panic and other realistic constraints. Through this thesis, experimental analysis for these sophisticated constraints are performed for both macro and micro agents in hopes for better understanding and formulating a safe egress for the agents.

Simulation tools have been used since the inception of computers and computation technology since the early 1970s. Modern simulations for evacuation and relief management includes the simulations of pedastrians and agents within the domains of the inscribed topology, with start and end points. These simulations are then run against various parameters and scenarios thus obtaining various results. Through this thesis the various experiments are simulated within the environment of PedSim, a massive microscopic real time simulator. The in depth analysis of large number agents based simulators are provided by the extensive and comprehensive survey presented in "Agent Based Modelling and Simulation tools: A review of the state-of-art software" by Sameera Abar et al. [22]. The survey is based on the following criteria which is presented as follows:

\begin{itemize}
  \item License/Availability
  \item Source Code in which the simulator is written in
  \item Type of Agent based on its interaction and behavior
  \item Coding language or application programming interface(API) for Model Development; Integrated Development Environment(IDE) used
  \item Compiler/Operating System/Implementation Platform
  \item Model Development Effort
  \item Modelling Strength/Simulation Models' Scalability Level
  \item ABMS scope or Application Domain
\end{itemize} 

From the extensive survey provided, it was easily inferred that PedSim, pedastrian microscopic simulator was the best suited for testing the various scenarios as it could support both micro and macro based agent modelling scenarios and grouping. It must also be mentioned that the tool is entirely written in C++, extremely light-weight and highly scalable. It also works well on development platforms and can be compliled easily on all operating systems. Its simple and easy to use, as complex simulations can be run for massive number of agents and the results can be visualized in real time. For all these reasons I have decided to use PedSim exclusively as my test bed tool whilst modelling such scenarios. 

In the following section the contribution and the implementation of my work will be discussed.

